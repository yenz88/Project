\documentclass[12pt,letterpaper]{article}

\usepackage{amsmath, amsthm, amssymb, amsfonts}
\usepackage{graphicx}
\usepackage{bm}
\usepackage{natbib}

\theoremstyle{definition}
\newtheorem{dfn}{Definition}

\begin{document}

% The numbers below controls the amount of space between the following sections
\def\shiftdowna{0.32in}  % Adjust for balance
\def\shiftdownb{0.22in}  % Adjust for balance

% Set up the boiler plate at the top of the page

\begin{center}
\textbf{{\large Project Work Statement}}\\


% SPONSOR
\vspace \shiftdowna
\underline {Sponsor}\\ 
\vspace{5pt}
\textbf{{\large McDonald's Corporation}}\\


% TITLE
\vspace \shiftdowna
\textbf{{\large How much Ice do You need?}}

% participants
\vspace \shiftdownb
\underline {Potential Participants}\\
\vspace{5pt}
Joyce Tan, \texttt{jtan21@jhu.edu} \\
\vspace{2pt}
Yen Theng Tan, \texttt{yen@jhu.edu} \\
\vspace{2pt}

% Mentor
\vspace{0.35in}
\vspace \shiftdownb
\underline {Academic Mentor} \\
\vspace{5pt}
\text{Nam Lee}, \texttt{nhlee@jhu.edu}

% DATE
\vspace \shiftdowna
Date: \today

\end{center}

\vfill  
%Fill page to force following note to bottom
\footnoterule
\noindent \small{Any apparent association of this work to McDonalds is
fictional one, and the sole purpose of this work is a class exercise}

\newpage

\section{Background} 
McDonald's Corporation is the world's largest chain of hamburger fastfood restaurants, serving around 68 million customers daily in 119 countries. Mcdonald's primarily sells hamburgers, cheeseburgers, chicken, French fries, breakfast items, soft drinks, milkshakes and desserts. In response to healthier consumer taste, the company has expanded its menu to include salads, wraps, smoothies and fruits.

\section{Problem Statement}

Selling soft drinks is a significant portion of McDonald's business, be it as a thirst quencher, or as part of the extra value meal. The server is not accustomed to putting much thought in measuring the amount of ice put in the cup. This often results in a overly diluted, overly concentrated or overly cold drink for the customer. This is likely to lower overall customer satisfaction, since a drink is a significant complement to a meal. Thus, customers are likely to appreciate if the right amount of ice was added for optimal satisfaction.
\\* To further define this problem, the exogenous variables are the proportion of ice to put in a drink. The endogenous variable would be the resulting temperature and concentration of the drink, as we are assuming that a customer's satisfaction is affected only by the temperature and concentration of the drink.

\section{Approach}
Firstly, we have a few assumptions on hand in order to simplify this problem. We assume that the consumer's taste depends entirely on the dilution and temperature of the drink. Also, any sample group that we use represents the population's preferences accurately.
\\* We are interested in approaching this problem using 2 different methods. The first method would be experimenting with different types of soda, and different amounts of ice to find out the optimal proportion of ice to soda. Using different proportions of ice, we will then measure the resulting temperature of the drink, as well as calculate the resulting dilution of the drink. We are also narrowing down the scope of our experiment to 4 of the most popular drinks in McDonalds' - Coca Cola, Sprite, Fanta Orange, and Diet Coke. By experimenting, we will test out which combination of temperature and dilution will yield the highest satisfaction from the test subjects. Over the course of 4 days, we will give the test subject 4 different cups of the same drink with different labels A, B, C, D. Additionally, we will take measurements four times a day, thereby including a time parameter of t=30seconds, 2 minutes, 5 minutes, t=infinity, which indicates the time elapsed after the ice is mixed with the drink. The different labels represent different ice proportions, and the test subject is allowed to sip the drink at time=t, assuming the ice is placed in the drink at t=0. On the same day, we will do the same test with the 3 other sodas, an hour apart. The cups are given an hour apart, so that the previous cup will not affect any judgement on the following drink. The test subject will then choose their favorite cup each round. We will tabulate the preferences of the entire sample group, and provide a conclusion about the consumer's preferred ice proportion in each soda drink.
\\* If time permits, we would be looking to approach from an alternative method. The second method would be using physics-based modeling. Utilizing the specific heat capacities of soda and ice (already found as specific values), we can calculate the different temperatures and dilution that the resulting drink will be. We can then compare this to the actual values obtained in the first approach, and see if they are pretty similar. This can also tell us more about the effect of the environment (heat loss to surrounding air and cup). This will be mainly a supporting tool and not used in place of the first approach.

\section{Milestones}
We have the following major deadlines:
\begin{itemize}
    \item Work Statement due date, October 24, 2012,
    \item Progress Report due date, Nov 5, 2012,
    \item Final Presentation due date, Nov 19, 2012,
    \item Final Report due date, Dec 3, 2012.
\end{itemize}

\section{Deliverable}
\subsection{From Team to Sponsor} % (fold)
The following outputs are expected from this project:
\begin{itemize}
    \item A table of optimal ice proportions for each different type of soda (namely Coca Cola, Sprite, Fanta Orange, Diet Coke),
    \item Matlab code with complete set of documentations that resulting temperature and dilution based on specific heat capacities and ice proportions,
    \item Numerical experiment results reporting success rate of different ice proportions,
    \item Technical report and presentations summarizing the work. 
\end{itemize}

\subsection{From Sponsor to Team} % (fold)

In order for our project to be of successful one, we will need:
\begin{itemize}
    \item Sufficient supply of the 4 different sodas we are concentrating on,
    \item Computing resources,
    \item Timely responses to inquiries.
\end{itemize}

%\newpage
%\bibliographystyle{plain}
%%\renewcommand\bibname{Selected Bibliography Including Cited Works}
%\nocite{*}
%\bibliography{biblio}

\end{document}
